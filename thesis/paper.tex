\documentclass[
	a4paper,
	english,
	twoside,
	openright,               
	11pt                            
	]{report}

\usepackage{textcomp}
\usepackage{listings}
\usepackage{pgfplots}

\usepackage[T1,OT1]{fontenc}
\usepackage[utf8]{inputenc}
\usepackage{subcaption}
\captionsetup{compatibility=false}
\usepackage{float}
\usepackage{chngcntr}
\usepackage{placeins}
\usepackage[ngerman,american]{babel}
\usepackage[toc,page]{appendix}
\usepackage{amssymb}
\usepackage{graphicx}
\usepackage{blindtext}
\usepackage{enumitem}
\usepackage{amsmath}
\usepackage[justification=centering]{caption}
\usepackage{glossaries}
\graphicspath{ {./img/} }
\PassOptionsToPackage{hyphens}{url}\usepackage{hyperref}
\setcounter{tocdepth}{3}
\pgfplotsset{compat=newest}

\begin{document}
%%%%%%%%%%%%%%%%%%%%%%%%%%%%%%%%%%%%%%%%%%%%%%%%%%%%%%%%%%%%%%%%%%%%%%
%%  Titelseite fuer eine Diplomarbeit/Dissertation an der Uni Wien  %%
%%                     zur Benutzung mit LaTeX                      %%
%%%%%%%%%%%%%%%%%%%%%%%%%%%%%%%%%%%%%%%%%%%%%%%%%%%%%%%%%%%%%%%%%%%%%%

%%  Erstellt anhand der Vorlagendefinition von
%%  http://www.univie.ac.at/Psychologie/cgi-bin/dbman/uploads/download/
%%      51_infoblatt__titelblatt_wissenschaftliche_arbeit.pdf
%%
%% Einzubinden in der eigenen document-Umgebung mittels \input{thesistitle}
%%
%% Ueberschriften wie "Titel der Diplomarbeit" oder "Verfasserin" (etc.)
%% muessen genau so stehen bleiben. Nur der Titel der Arbeit oder die Namen
%% sind entsprechend beliebig.

% Stephan Paukner, 14.08.2007
% Obige Vorlagendefinition schlaegt Arial oder eine vergleichbare serifenlose
% Schriftart vor. Der serifenlose Font von LaTeX, CMSS, hat leider keine
% fette (bold) Variante. Arial ist allerdings ein kommerzieller Font und unter
% LaTeX standardmaessig nicht verfuegbar. Am naehesten kommt dem der Font
% Helvetica, einzubinden via \usepackage[scaled=0.90]{helvet} in der Praeambel.

\begin{titlepage}
\vspace*{-2cm}  % bei Verwendung von vmargin.sty
\begin{flushright}
	\includegraphics[width=.5\textwidth]{titlepage/RZ_Logo_Uni_sw_01.jpg}
\end{flushright}
\vspace{1cm}

\begin{center}  % Diplomarbeit ODER Magisterarbeit ODER Dissertation
    \huge{\textbf{\textsf{\MakeUppercase{
        MASTERARBEIT / MASTER'S THESIS
    }}}}
    \vspace{2cm}

    \large{\textsf{  % Diplomarbeit ODER Magisterarbeit ODER Dissertation
                     % (Dies ist erst die Ueberschrift!)
        Titel der Masterarbeit / Title of the Master's Thesis
    }}
    \vspace{.1cm}

    \LARGE{\textsf{  % Hier kommt der eigentliche Titel, bei Bedarf mit \\
                     % ACHTUNG: Deutsche Anfuehrungszeichen: ,,Titel``
                     %          English quotes:              ``title''
        % >>>>> BEGINN TITEL >>>>>
        {\glqq}Tool for Visual Cluster Analysis and Consensus Clustering\grqq
        %``The Creation of a Title Page''
        % <<<<< ENDE TITEL <<<<<
    }}
    \vspace{2cm}

    \large{\textsf{  % Verfasserin ODER Verfasser (Ueberschrift)
        verfasst von / submitted by
    }}

    \Large{\textsf{  % Vorname Nachname
        % >>>>> BEGINN VORNAME NACHNAME >>>>>
        Christian Permann, BSc
        % <<<<< ENDE VORNAME NACHNAME <<<<<
    }}
    \vspace{2cm}

    \large{\textsf{
        angestrebter akademischer Grad / in partial fulfilment of the requirements for the degree  % (Ueberschrift)
    }}

    \Large{\textsf{  % Magistra ODER Magister ODER Doktorin ODER Doktor
                     % ACHTUNG: Kuerzel "Mag.a" oder "Dr.in" nicht zulaessig
				Master of Science (MSc)
    }}
\end{center}
\vspace{2cm}

\noindent\textsf{Wien, 2020 / Vienna, 2020}  % <<<<< ORT, MONAT UND JAHR EINTRAGEN
\vfill

\noindent\begin{tabular}{@{}p{8cm}p{8cm}}
\textsf{Studienkennzahl lt. Studienblatt / }
&
\textsf{UA 066 921} % <<<<< STUDIENKENNZAHL LT. CURRICULUM EINTRAGEN
\\
\textsf{degree programme code as it appears on}
&
\\
\textsf{the student record sheet}
&
\\
\end{tabular}

\vspace{0.2cm}
    % BEI DISSERTATIONEN:
%\textsf{Dissertationsgebiet lt. Studienblatt:}
    % ANSONSTEN:
\noindent\begin{tabular}{@{}p{8cm}p{8cm}}
\textsf{Studienrichtung lt.\ Studienblatt:/}
&
\textsf{Masterstudium Informatik}  % <<<<< DISSGEBIET/STUDIENRICHTUNG EINTRAGEN
\\
\textsf{degree programme as it appears on}
&
\\
\textsf{the student record sheet}
&
\\
\end{tabular}

\vspace{0.2cm}

\noindent\begin{tabular}{@{}p{8cm}p{8cm}}
% Betreuerin ODER Betreuer:
\textsf{Betreut von / Supervisor:}
&
\textsf{Univ.-Prof. Dipl.-Inform.Univ. Dr. Claudia Plant}  % <<<<< NAME EINTRAGEN
\\
%&
%\textsf{2. Betreuer mit Titel}  % <<<<< NAME EINTRAGEN
%\textsf{degree(s) first name family name}
\end{tabular}

\end{titlepage}


% \newpage
%
% \thispagestyle{empty}
% \text{ }
% \vspace{13.5cm}
%
%
% Martin Perdacher\\
% Teybergasse 4/5\\
% 1140 Wien\\
%
% Hiermit versichere ich, dass ich die von mir vorgelegte Arbeit selbstständig verfasst habe, dass ich die verwendeten Quellen, Internet-Quellen und Hilfsmittel vollständig angegeben habe und dass ich die Stellen der Arbeit -- einschließlich Tabellen, Karten und Abbildungen~--, die anderen Werken oder dem Internet im Wortlaut oder dem Sinn nach entnommen sind, auf jeden Fall unter Angabe der Quelle als Entlehnung kenntlich gemacht habe.\\

% Wien, den 1. Dezember 2020\\
% \medskip
% \medskip
%
% (Unterschrift)\\
% \underline{~~~~~~~~~~~~~~~~~~~~~~~~~~~~~~~~~~~~~~~~}\\
% Martin Perdacher\\

%\newpage



\chapter*{Danksagungen} % (fold)
\label{cha:acknowledgements}
\noindent

Ich möchte meiner Betreuerin Univ.-Prof. Dipl.-Inform.Univ. Dr. Claudia Plant für Ihre Unterstützung bei dieser Arbeit danken. Unsere offenen und konstruktiven Gespräche haben mir sehr geholfen diese Arbeit zielstrebig und motiviert zu vollenden.

INCOMPLETE
% chapter acknowledgements (end)


\pagenumbering{gobble}

\tableofcontents
 \cleardoublepage
%
%
% list of figures
\listoffigures
%\protect addcontentsline{toc}{chapter}{Abbildungsverzeichnis}
\cleardoublepage

% list of tables
\listoftables
%\protect addcontentsline{toc}{chapter}{Tabellenverzeichnis}
\cleardoublepage

% list of algorithms
%\listofalgorithmsLeMATo
%\protect addcontentsline{toc}{chapter}{Algorithmenverzeichnis}
\cleardoublepage

 \part{Introduction and Preliminaries}
 \pagenumbering{arabic}
   \setcounter{page}{1}
 \iffalse \bibliography{../references.bib} \fi

\chapter{Introduction}
\label{cha:Introduction}

The generation of clusterings is quite a difficult task when little knowledge on the data is available. There exist a large number of clustering algorithms that all aim to find a fitting partitioning of the data, but they all come with different assumptions on the data. Taking k-Means as an example, the data is assumed to be Gaussian distributed which might not actually be the case for many data-sets, making it unsuitable for clustering such data. 

As a solution for this, visual frameworks have been proposed that aim to help finding the right clustering by evaluating the results of different algorithms and parameter settings via quality metrics and ranking the clustering solutions. One of these frameworks is Clustervision \cite{Kwon2018ClustervisionVS} whereby different algorithms are run, their solutions are visualized and ranked according to the average of five different quality metrics. The user is then able to look at the candidates through different visualizations and choose a solution as final result. A problem with this approach is though, that each metric is biased towards a specific resulting cluster structure, possibly ranking it higher than others even if it does not fit the data properly. Even the authors of Clustervision themselves mention that ``the effectiveness of these metrics in gauging the quality of the clustering is also difficult to determine due to the lack of ground truth''. Also, as many solutions were calculated and only one of them is chosen in the final decision, a lot of calculated knowledge is ignored.

To overcome this problem, I propose to not use quality metrics for the choice of the best clustering, but to rely on robustness indicators for this selection. As robustness is difficult to evaluate by itself many different algorithms with different parameters should be used, analyzing where multiple methods or parameter sets find a similar result. Clusterings that are similar to many other results can be seen as robust and finding a consensus of those can result in a better overall clustering. To find such similar clusterings or groups of similar clusterings meta-clustering can be used. To then compute final candidates, each group of similar results can be merged to a consensus result, capturing the essence of the group. 

Based on this idea I created a tool which will be further described in Chapter \ref{cha:Tool}. Additionally, the tool aims to include the expert user into the process of finding the result. Most research on consensus clustering, as also seen in the survey of Vega-Pons and Ruiz-Shulcloper \cite{survey1}, only briefly mention how generating or pre-selecting base clusterings can impact the result of consensus clustering. For this, my tool allows the user to visually explore the clusterings created by the simple clustering algorithms, facilitating the choice of which ones to merge for computing a final result.

The rest of this thesis is organized as follows: [TEXT]

% TODO: describe implementation section and evaluation section textually



\part{Theory}

\chapter{Related Work}\label{cha:related_work}
\section{Clustering Algorithms}

A lot of research as been put toward the topic of clustering, resulting in a large number of different methods. \cite{surveyclustering}

\section{Visual Clustering Frameworks}

Clustervision \cite{Kwon2018ClustervisionVS}

VISTA \cite{VISTA} ClusterMap and user result evaluation

iVIBRATE \cite{10.1145/1148020.1148024} extension VISTA?

simple visualizations ELKI implementation \cite{10.14778/2824032.2824115}


Frameworks with specific data-sets in mind:

Propose: ``An Evolutionary and Visual Framework for Clustering of DNA Microarray Data'' \cite{DNAVis}, tool for clustering DNA data with visualization

Even Memes \cite{Dang2017}

\section{Consensus Clustering}

consensus cluster plus \cite{10.1093/bioinformatics/btq170}

\chapter{Methods}

\section{Clustering}
\subsection{OPTICS}

\section{Consensus Clustering}
\subsection{DICLENS}

\section{Other Methods}
\subsection{Hungarians Method}

\part{Implementation}

\chapter{Implementation Remarks}

\section{Programming Language and Tools}

For this implementation my language of choice was Java. With Java there are lots of available implementations for different clustering methods available, including the Frameworks ELKI [CITATION] and WEKA [CITATION]. Additionally, it is easy to create custom visualizations using the low level draw() method of JComponents in Swing. To improve the performance in some parts of the tool Java 1.8 was chosen as lowest required version, as it allows to use of parallel streams to easily implement parallelism.

\section{Interfaces and Extendability}

\section{Clustering Algorithms}

\section{Meta-Clustering}

\section{Consensus Clustering}

\chapter{The Tool}\label{cha:Tool}

\section{I/O and Data Visualization}
bla

\section{Selection of Algorithms and Parameters}
bla

\section{Meta-View and Consensus Clustering}
bla

\part{Testing}
\chapter{Experiments}\label{cha:experiments}
bla

\part{Concluding Thoughts}

\chapter{Future Work}\label{cha:futurework}

\section{Improvements for Tool}

bla

\section{Research Consensus Clustering}

bla

\section{Visual Frameworks}

bla


\chapter{Conclusion}\label{cha:conclusion}

\section{Lessons learned}
bla 

\section{Reflection of Work}
bla

\begin{appendices}

\chapter{Abstract} % (fold)
\label{cha:abstract}

\section{English abstract} % (fold)
\label{sec:english_abstract}

\iffalse \bibliography{../references.bib} \fi

%\begin{abstract}
Finding a good clustering solution for an unexplored data-set is a non-trivial task. Due to the large number of clustering algorithms that usually have lots of parameters, clustering results may differ strongly from each other and the underlying ground truth. With only little knowledge on the data the evaluation of which result best represents the underlying cluster structure is difficult. To find a fitting selection for this choice, different visual frameworks exist that aim to simplify this choice, usually by ranking the results according to quality measures. As those measures also have the downside of being biased towards specific structures (whether or not they fit the data) they are problematic for selecting a final result. For this reason, I propose to purely use indicators of robustness for the creation or selection of a clustering result. This is done by meta-clustering results from different clustering algorithms and calculating consensus clusterings from groups of similar clusterings. Additionally, this process is supported through visualizations, giving the expert user the possibility to use his knowledge to further improve on the final result.
%\end{abstract}


% section english_abstract (end)
\newpage

\section{Deutsche Zusammenfassung} % (fold)
\label{sec:deutsche_zusammenfassung}

\iffalse \bibliography{../references.bib} \fi

\begin{otherlanguage}{ngerman}

%\begin{abstract}
Eine gute Clustering Lösung für wenig erforschte Daten zu finden ist eine komplexe Aufgabe. Wegen der großen Anzahl an Clustering Algorithmen, welche meist auch viele verschiedene Parameter benötigen, können sich die Ergebnisse stark untereinander, aber auch von dem richtigen Ergebnis, unterscheiden. Mit nur wenig Wissen über die Daten ist auch die Evaluierung welches Ergebnis am nähersten zu der der unterliegenden Wahrheit, beziehungsweise am besten der Struktur der Daten entspricht eine schwere Aufgabe.  Um eine solche Auswahl besser treffen zu können wurden visuelle Frameworks erschaffen, die meist mittels Qualitäts-Metriken die verschiedenen Ergebnisse bewerten und gereiht anzeigen. Da diese Metriken aber auch das Problem haben gewisse Strukturen in Ergebnissen zu bevorzugen zeigen sie sich wiederum bei der Entscheidung über das endgültige Ergebnis als problematisch. Aus diesem Grund schlage ich vor die Eigenschaft wie robust ein Ergebnis ist für die finale Entscheidung heranzuziehen. Um dies zu tun werden die Clusterings auf Meta-Ebene nochmals geclustert, wobei ähnliche Ergebnisse in einer Gruppe mittels Consensus Clustering zu einer Lösung zusammengeführt werden. Dieser Prozess wird weiters durch Visualisierungen unterstützt, so dass ein Experte mit Hilfe seines Wissens die Lösung möglicherweise noch weiter verbessern kann.
%\end{abstract}

\end{otherlanguage}


% section deutsche_zusammenfassung (end)

% section abstract (end)


% chapter abstract (end)


% \include{appendix/acronyms}
% \include{appendix/userGuide}
% \include{appendix/supplementary}

\end{appendices}

\cleardoublepage

\bibliographystyle{acm} 

\bibliography{references}

\printglossaries


\end{document}
