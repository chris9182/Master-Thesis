\iffalse \bibliography{../references.bib} \fi

%\begin{abstract}
Finding a good clustering solution for an unexplored data-set is a non-trivial task. Due to the large number of clustering algorithms which usually have lots of parameters, clustering results may differ strongly from each other and the underlying ground truth. With only little knowledge on the data the evaluation of which result best represents the underlying cluster structure is difficult. To find a fitting selection for the result, there exist visual frameworks that aim to simplify this choice by ranking the results according to quality measures. As those measures also have the downside of being biased towards specific structures (weather or not they fit the data) they are problematic for selecting a final result. For this reason, I propose to purely use indicators of robustness for the creation of a clustering result. This is done by meta-clustering results from different clustering algorithms and results and calculating consensus clusterings from each group of simmilar results. Additionally this process is supported through visualizations, giving the expert user the possibility to use his knowledge to further improve on the final result.
%\end{abstract}
