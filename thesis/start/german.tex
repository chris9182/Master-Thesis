\iffalse \bibliography{../references.bib} \fi

\begin{otherlanguage}{ngerman}

%\begin{abstract}
Eine gute Clustering Lösung für wenig erforschte Daten zu finden ist eine komplexe Aufgabe. Wegen der großen Anzahl an Clustering Algorithmen, welche meist auch viele verschiedene Parameter benötigen, können sich die Ergebnisse stark untereinander, aber auch von dem richtigen Ergebnis, unterscheiden. Mit nur wenig Wissen über die Daten ist auch die Evaluierung welches Ergebnis am nähersten zu der der unterliegenden Wahrheit, beziehungsweise am besten der Struktur der Daten entspricht eine schwere Aufgabe.  Um eine solche Auswahl besser treffen zu können wurden visuelle Frameworks erschaffen, die meist mittels Qualitäts-Metriken die verschiedenen Ergebnisse bewerten und gereiht anzeigen. Da diese Metriken aber auch das Problem haben gewisse Strukturen in Ergebnissen zu bevorzugen zeigen sie sich wiederum bei der Entscheidung über das endgültige Ergebnis als problematisch. Aus diesem Grund schlage ich vor die Eigenschaft wie robust ein Ergebnis ist für die finale Entscheidung heranzuziehen. Um dies zu tun werden die Clusterings auf Meta-Ebene nochmals geclustert, wobei ähnliche Ergebnisse in einer Gruppe mittels Consensus Clustering zu einer Lösung zusammengeführt werden. Dieser Prozess wird weiters durch Visualisierungen unterstützt, so dass ein Experte mit Hilfe seines Wissens die Lösung möglicherweise noch weiter verbessern kann.
%\end{abstract}

\end{otherlanguage}
