%%%%%%%%%%%%%%%%%%%%%%%%%%%%%%%%%%%%%%%%%%%%%%%%%%%%%%%%%%%%%%%%%%%%%%
%%  Titelseite fuer eine Diplomarbeit/Dissertation an der Uni Wien  %%
%%                     zur Benutzung mit LaTeX                      %%
%%%%%%%%%%%%%%%%%%%%%%%%%%%%%%%%%%%%%%%%%%%%%%%%%%%%%%%%%%%%%%%%%%%%%%

%%  Erstellt anhand der Vorlagendefinition von
%%  http://www.univie.ac.at/Psychologie/cgi-bin/dbman/uploads/download/
%%      51_infoblatt__titelblatt_wissenschaftliche_arbeit.pdf
%%
%% Einzubinden in der eigenen document-Umgebung mittels \input{thesistitle}
%%
%% Ueberschriften wie "Titel der Diplomarbeit" oder "Verfasserin" (etc.)
%% muessen genau so stehen bleiben. Nur der Titel der Arbeit oder die Namen
%% sind entsprechend beliebig.

% Stephan Paukner, 14.08.2007
% Obige Vorlagendefinition schlaegt Arial oder eine vergleichbare serifenlose
% Schriftart vor. Der serifenlose Font von LaTeX, CMSS, hat leider keine
% fette (bold) Variante. Arial ist allerdings ein kommerzieller Font und unter
% LaTeX standardmaessig nicht verfuegbar. Am naehesten kommt dem der Font
% Helvetica, einzubinden via \usepackage[scaled=0.90]{helvet} in der Praeambel.

\begin{titlepage}
\vspace*{-2cm}  % bei Verwendung von vmargin.sty
\begin{flushright}
	\includegraphics[width=.5\textwidth]{titlepage/RZ_Logo_Uni_sw_01.jpg}
\end{flushright}
\vspace{1cm}

\begin{center}  % Diplomarbeit ODER Magisterarbeit ODER Dissertation
    \huge{\textbf{\textsf{\MakeUppercase{
        MASTERARBEIT / MASTER'S THESIS
    }}}}
    \vspace{2cm}

    \large{\textsf{  % Diplomarbeit ODER Magisterarbeit ODER Dissertation
                     % (Dies ist erst die Ueberschrift!)
        Titel der Masterarbeit / Title of the Master's Thesis
    }}
    \vspace{.1cm}

    \LARGE{\textsf{  % Hier kommt der eigentliche Titel, bei Bedarf mit \\
                     % ACHTUNG: Deutsche Anfuehrungszeichen: ,,Titel``
                     %          English quotes:              ``title''
        % >>>>> BEGINN TITEL >>>>>
        {\glqq}Tool for Visual Cluster Analysis and Consensus Clustering\grqq
        %``The Creation of a Title Page''
        % <<<<< ENDE TITEL <<<<<
    }}
    \vspace{2cm}

    \large{\textsf{  % Verfasserin ODER Verfasser (Ueberschrift)
        verfasst von / submitted by
    }}

    \Large{\textsf{  % Vorname Nachname
        % >>>>> BEGINN VORNAME NACHNAME >>>>>
        Christian Permann, BSc
        % <<<<< ENDE VORNAME NACHNAME <<<<<
    }}
    \vspace{2cm}

    \large{\textsf{
        angestrebter akademischer Grad / in partial fulfilment of the requirements for the degree  % (Ueberschrift)
    }}

    \Large{\textsf{  % Magistra ODER Magister ODER Doktorin ODER Doktor
                     % ACHTUNG: Kuerzel "Mag.a" oder "Dr.in" nicht zulaessig
				Master of Science (MSc)
    }}
\end{center}
\vspace{2cm}

\noindent\textsf{Wien, 2020 / Vienna, 2020}  % <<<<< ORT, MONAT UND JAHR EINTRAGEN
\vfill

\noindent\begin{tabular}{@{}p{8cm}p{8cm}}
\textsf{Studienkennzahl lt. Studienblatt / }
&
\textsf{UA 066 921} % <<<<< STUDIENKENNZAHL LT. CURRICULUM EINTRAGEN
\\
\textsf{degree programme code as it appears on}
&
\\
\textsf{the student record sheet}
&
\\
\end{tabular}

\vspace{0.2cm}
    % BEI DISSERTATIONEN:
%\textsf{Dissertationsgebiet lt. Studienblatt:}
    % ANSONSTEN:
\noindent\begin{tabular}{@{}p{8cm}p{8cm}}
\textsf{Studienrichtung lt.\ Studienblatt:/}
&
\textsf{Masterstudium Informatik}  % <<<<< DISSGEBIET/STUDIENRICHTUNG EINTRAGEN
\\
\textsf{degree programme as it appears on}
&
\\
\textsf{the student record sheet}
&
\\
\end{tabular}

\vspace{0.2cm}

\noindent\begin{tabular}{@{}p{8cm}p{8cm}}
% Betreuerin ODER Betreuer:
\textsf{Betreut von / Supervisor:}
&
\textsf{Univ.-Prof. Dipl.-Inform.Univ. Dr. Claudia Plant}  % <<<<< NAME EINTRAGEN
\\
%&
%\textsf{2. Betreuer mit Titel}  % <<<<< NAME EINTRAGEN
%\textsf{degree(s) first name family name}
\end{tabular}

\end{titlepage}


% \newpage
%
% \thispagestyle{empty}
% \text{ }
% \vspace{13.5cm}
%
%
% Martin Perdacher\\
% Teybergasse 4/5\\
% 1140 Wien\\
%
% Hiermit versichere ich, dass ich die von mir vorgelegte Arbeit selbstständig verfasst habe, dass ich die verwendeten Quellen, Internet-Quellen und Hilfsmittel vollständig angegeben habe und dass ich die Stellen der Arbeit -- einschließlich Tabellen, Karten und Abbildungen~--, die anderen Werken oder dem Internet im Wortlaut oder dem Sinn nach entnommen sind, auf jeden Fall unter Angabe der Quelle als Entlehnung kenntlich gemacht habe.\\

% Wien, den 1. Dezember 2020\\
% \medskip
% \medskip
%
% (Unterschrift)\\
% \underline{~~~~~~~~~~~~~~~~~~~~~~~~~~~~~~~~~~~~~~~~}\\
% Martin Perdacher\\

%\newpage
